The more our society depends on digital systems, the more important private 
communication becomes.
We need private communications to sustain democracy, thus we need it to be 
available to everyone.
The purpose of this laboratory work is to introduce some practical aspects of 
private messaging.
More specifically, after it, you should be able to
\begin{itemize}
  \item \emph{apply} (securely!) some common implementations of cryptography 
    for private communication --- also including any set-up (e.g.\ key 
    verification).
  \item \emph{analyse} different systems for private communication based on 
    their security properties and \emph{evaluate} which is suitable in a given 
    situation.
  \item \emph{evaluate} different implementations of private communication from 
    a usability perspective.
\end{itemize}

This assignment treats the following topics: usability, symmetric-key and 
public-key cryptography and steganography.
The usability is covered in Chapter 2 of 
\citetitle{Anderson2008sea}~\cite{Anderson2008sea}.
The cryptography parts are covered in Chapter 5 of the same 
book~\cite{Anderson2008sea}.
Section 23.4 in~\cite{Anderson2008sea} treats privacy enhancing technologies 
and is also of importance to us.
Finally, steganogrphy is also treated in 
\citetitle{johnson1998exploring}~\cite{johnson1998exploring}.
(Other recommended papers are 
\citetitle{anderson1998limits}~\cite{anderson1998limits} and 
\citetitle{provos2003hide}~\cite{provos2003hide}.)

During this assignment you should consult the 
documentation~\cite{gpgdoc,gpg4windoc,outguess,openpuffdoc} for instructions on
how to use the specific softwares.
