\mode*

\section{Introduction}%
\label{Introduction}

In this lab we will evaluate some tools for private communication, in 
particular, email and instant messaging.

Email security has the longest history, this has been around since the 80s.
Have they gained wide-spread adoption yet?
No.
There are two competing approaches: Pretty Good Privacy (PGP) and S/MIME.
But there are alternatives coming, \eg ProtonMail made the headlines a few 
years ago.

The security of instant messaging looks better than that of email.
This is largely due to the extensive use of smartphones for messaging.
There are apps such as Signal, WhatsApp (uses Signal's protocol) and a few 
more.
Have they gained wide-spread adoption yet?
More than for email.


\section{Assignment}%
\label{sec:Tasks}

The assignment is divided into two parts: first, a preparation part and, 
second, the seminar part.

\subsection{Before the seminar}

Before starting, read Chapter 23.4 of 
\citetitle{Anderson2008sea}~\cite{Anderson2008sea}.
Work in your study groups to prepare for the seminar.

\paragraph{Email security}

List the security (and privacy) expectations that you have on email.
Try to install and use a tool (\eg GPG, ProtonMail, MixMaster, MixMinion \etc) 
that achieves those expectations, or comes as close as possible, for securing 
an email conversation with a friend.

\paragraph{Instant-messaging security}

List the security (and privacy) expectations that you have on instant messages.
(Are they different than for email?)
Try to install and use a tool (\eg Signal, WhatsApp, \etc) that achieves those 
expectations, or comes as close as possible, for securing a conversation with a 
friend.

\paragraph{Reflection}

Discuss the following questions in the group:
\begin{itemize}
  \item How do these systems work?
  \item How do you use them securely?
  \item Did the tools fulfil your requirements?
  \item What were the biggest problems that you encountered?
    What must be done to resolve them?
    Can they be resolved?
\end{itemize}

Read the \enquote{Why Johnny can't encrypt} papers which has studied exactly 
this problem throughout the years:
\begin{itemize}
  \item \citetitle{WhyJohnnyCantEncrypt}~\cite{WhyJohnnyCantEncrypt},
  \item \citetitle{WhyJohnnyStillCantEncrypt}~\cite{WhyJohnnyStillCantEncrypt},
  \item \citetitle{WhyJohnnyStillStillCantEncrypt}~\cite{WhyJohnnyStillStillCantEncrypt},
\end{itemize}
all concern email, whereas 
\citetitle{CanJohnnyFinallyEncrypt}~\cite{CanJohnnyFinallyEncrypt} concerns 
instant messaging.
What problems did the researchers find?
What methods did they use?
Discuss these questions in the group.

\subsection{During the seminar}

We will divide the time of the seminar into three parts.

\paragraph{What tools and how did they work?}

We will then take 20 minutes to discuss (in groups) what tools you tried and 
the usability of these tools.
Discuss the following questions:
\begin{itemize}
  \item What tools did you try?
    How does these work and, more importantly, how do you use them securely?

  \item How easy was it to use them (securely)?
    How did you evaluate them?
\end{itemize}

The groups will then summarize the key points of their discussion (\eg what was 
most interesting) in class.
We will dedicate 20 minutes to this.

\paragraph{What did the papers say?}

We will take the first 20 minutes to discuss (in groups) the papers that you 
read for the seminar.
Discuss the following questions:
\begin{itemize}
  \item What were the research questions of the papers?
  \item What methods did the researchers use to answer the research questions?
  \item What is the main take-away message?
\end{itemize}

We will then take 15 minutes for the groups to summarize their discussions in 
class.

\paragraph{Future outlook}

We will conclude the seminar with a future outlook.
Based on your discussions:
\begin{itemize}
  \item What were the biggest problems?
  \item Can we resolve them, how?
  \item What is the way forward?
\end{itemize}


\section{Examination}

Note down your thoughts from reading, experimenting and the reflections for the 
questions above and bring to the seminar session.


\printbibliography
