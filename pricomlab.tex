\documentclass[a4paper]{llncs}
\usepackage[utf8]{inputenc}
\usepackage[T1]{fontenc}
\usepackage[swedish,english]{babel}
\usepackage[hyphens]{url}
\usepackage{hyperref}
\usepackage{cleveref}
\usepackage{varioref}

%\usepackage[natbib,style=lncs,maxbibnames=99]{biblatex}
\usepackage[natbib,style=numeric-comp,maxbibnames=99]{biblatex}
\addbibresource{pricomlab.bib}

\title{Lab: Private Communication}
\author{%
  Daniel Bosk
  \and
  Lennart Franked
}
\institute{%
  Department of Information and Communication Systems\\
  Mid Sweden University, SE-851\,70 Sundsvall\\
}
\date{\today}

\begin{document}
\maketitle


\section{Introduction}
\label{Introduction}
This laboratory assignment will cover how public and private keys are used in 
practice, how to use these to encrypt and decrypt as well as sign and verify 
a message or a file, also some basic key distribution.
We will also cover alternative ways to try to evade prying eyes from finding
your message.
We will use the open source programs GNU Privacy Guard (GPG) and OpenPuff.


\section{Aim}
\label{sec:Aim}
After completion of this assignment you will
\begin{itemize}
    \item Have an understanding of how to use a public--private key-pair.
\item Know how to use implementations of asymmetric ciphers.
\item Be able to distribute your own key and retrieve other public keys using 
publicly available key servers.
\item Be able to use steganography as a way to hide messages.

\end{itemize}


\section{Reading instructions}
\label{sec:Reading}
Before starting this assignment you should have read chapters 5 and 23.4.4--5 
in \citetitle{Anderson2008sea}~\cite{Anderson2008sea}.
You should also read the paper 
\citetitle{johnson1998exploring}~\cite{johnson1998exploring} to fully 
understand how steganography works in practice.
(Other recommended papers are 
\citetitle{anderson1998limits}~\cite{anderson1998limits} and 
\citetitle{provos2003hide}~\cite{provos2003hide}.)

During this assignment you should consult the 
documentation~\cite{gpgdoc,gpg4windoc,outguess,openpuffdoc} for instructions on
how to use the specific softwares.



\section{Tasks}
\label{sec:Tasks}
This assignment is divided into two parts.
The first part will cover cryptography---to hide the data---using email as 
a usage example.
The second part will cover steganography---to hide the presence of data.

\subsection{Email security}
\label{subsec:Email}
In this part you will work with email security.
You will start by creating your own key-pair after which you will upload your 
public key to one of the public-key servers.
Once you have done that you will send an encrypted email to one of your
classmates which will be your lab partner.
You should be able to find his or her public key on the key servers.

Start by downloading and installing GPG,\footnote{%
  If you are in a computer lab on campus, this is most probably already done.
} select the appropriate version of GPG depending on what operating system you 
use.
Once you have GPG installed on your system, generate a key-pair and \emph{make 
sure to make an active choice for what cipher and key size to use}.
When finished, export your public key to the following key server:
\begin{center}
  \url{pgp.mit.edu}
\end{center}

Next you shall import both your lab partner's key and your tutor's key, they
should be available on the same key server.
When you and your lab partner have each other's public keys, send an encrypted
email to each other and \emph{confirm that the other party is able to decrypt 
your email}.

Next, find a way to communicate with your partner, such that you can
confirm that they are who they say they are, and ask them to repeat the 
fingerprint of their public key.
Once verified you can publicly sign their public key with an approprite trust 
level based on the type of verification you did.
Give this signed version of the key back to your lab partner so he or she can 
import it and resend it to the key server.

When you have completed these steps you must send an encrypted email to the 
tutor and await an encrypted response.
The tutor will not accept your email unless your key on the key server is 
signed by at least one person.

\subsubsection{Social engineering using spoofing (optional)}
\label{subsec:Social}
Try to trick a different classmate that you are their lab partner.
Do this by spoofing an email or any other form of communication.
Will you be able trick them into believing that your fingerprint is their 
partners?
What measures must be taken in order not to be tricked?

\subsection{Steganography}
\label{subsec:Steganograhy}
In this part you will work with steganography.
By using either Outguess or OpenPuff you will get a practical understanding of 
how steganography works. 

Outguess is already installed in the computer lab.
But it should be available in most BSD and Linux repositories.
You can find OpenPuff at the following URL\@:
\begin{center}
  \url{http://embeddedsw.net/OpenPuff_Steganography_Home.html}
\end{center}

Once you have either program installed, write a message in a text file and hide 
it in a picture.
You will then post this picture in the course forum, where your partner can
access your picture and retrieve the hidden message.
You can use a password with the steganographic software, encrypt this password 
for your lab partner's and the tutor's key using GPG\@.
Post this encrypted secret in the course forum with the picture.
Now only your lab partner and the tutor will be able to decrypt the secret and 
retrieve the message from the picture.

\subsubsection{Retrieving other groups' messages (Optional)}
\label{subsec:Stegopt}
Try to retrieve a hidden message posted by another group.
If retrieved, post the hidden message as a reply to that picture.
You will then be eligible for the grade \emph{Pass with Distinction} on this 
assignment.

% \subsection{A more rewarding assignment}
% 
% This assignment is for the more ambitious students, those who are eager to 
% learn more.
% This assignment will take you on an adventure where you will have to learn a 
% lot more to be able to finish the tasks.
% It is not a waste of time, learning never is, it will reward you in terms of 
% knowledge.
%
% To start this assignment: ``The HTTP server address lies within the hidden 
% key''.
%
% Best of luck!


\section{Examination}
\label{sec:Exam}
You must hand in a report including the following:
\begin{itemize}
  \item A short summary of your key-generation process, including what cipher 
    you chose and the key length.
    \emph{You must motivate this, ``Because it was default'' is not a valid 
    motivation.}
  \item An explanation detailing how you verified that your partners public key 
    was the correct one.
    Motivate why you chose the method you did, and based on this, how certain 
    you are that this person actually is the same person that owns the public 
    key.
  \item A copy of the secret message you received from your tutor by email.
  \item A 250-words detailed description of how steganography works.
  \item Any proof of that you solved the optional assignments.

%  \item We will know if you solved the \enquote{more rewarding} assignment.
%    Why will be obvious once you solve the assignment.
\end{itemize}


\printbibliography{}
\end{document}
